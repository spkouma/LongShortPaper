\documentclass[journal=mamobx,manuscript=article]{achemso}


\usepackage[version=3]{mhchem}
\usepackage{amsmath}
\usepackage{xcolor}
\usepackage{amsfonts}
\usepackage{amssymb}
\usepackage{graphicx}
\usepackage[english]{babel}
\usepackage{chemfig}
\newcommand{\species}[1]{\textit{#1} sp.}
\usepackage{subcaption}
\usepackage{cancel}




%%%%%%%%%%%%%%%%%%%%%%%%%%%%%%%%%%%%%%%%%%%%%%%%%%%%%%%%%%%%%%%%%%%%%
%% If issues arise when submitting your manuscript, you may want to
%% un-comment the next line.  This provides information on the
%% version of every file you have used.
%%%%%%%%%%%%%%%%%%%%%%%%%%%%%%%%%%%%%%%%%%%%%%%%%%%%%%%%%%%%%%%%%%%%%
%%\listfiles



\newcommand*\mycommand[1]{\texttt{\emph{#1}}}


\author{Sperydon Koumarianos}
\affiliation{Department of Physics and Astronomy, York University, Toronto, ON, Canada. M3J 1P3} 
\alsoaffiliation{Department of Mathematics, York University, Toronto, ON, Canada.  M3J 1P3}

 
\author{Rohith Kaiyum}
\affiliation{Department of Physics and Astronomy, York University, Toronto, ON, Canada. M3J 1P3} 

\author{Christopher J. Barrett}
\affiliation{Department of Chemistry, McGill University, Montreal, QC, Canada.  H3A 2K6}

\author{Neal Madras}
\affiliation{Department of Mathematics, York University, Toronto, ON, Canada.  M3J 1P3}

\author{Ozzy Mermut}
\affiliation{Department of Physics and Astronomy, York University, Toronto, ON, Canada. M3J 1P3}
\email{omermut@yorku.ca}


\title[An \textsf{achemso} demo]
  {Effect of Chain Length on the Layered Adsorption of Polyelectrolytes to Surfaces: Theory and Experiment}


\keywords{American Chemical Society, \LaTeX}

\begin{document}

\begin{abstract}

\end{abstract}

\section{2. Introduction}

In the interest of developing new molecular architectures that cover a wide range of length scales, various methods to self-assemble and organize polymers on surfaces continue to be explored, from covalent attachment techniques\cite{doi:10.1002/ijch.199600050,B210143M,doi:10.1002/masy.200450305} to physisorption approaches.\cite{Chen1992,Serizawa2002}  Recently, a new technique for preparing polymer thin films which uses electrostatic interactions between oppositely charged polyelectrolytes to assemble a multilayer organic film, has attracted attention from broad areas of materials science.\cite{Decher1997}  Polyelectrolyte multilayer films (PEMs) are built-up layer-by-layer (LBL) by alternating adsorption of polycations and polyanions from aqueous solution onto a surface.\cite{Decher2006}  Growing interest in these self-assembled films can in part be attributed to the fact that their fabrication is experimentally simple, and versatile with respect to the surfaces and materials that can be utilized.  Furthermore, industrial applications of functional polyelectrolyte multilayer films (PEMs), for example as biomedical materials, have given impetus to explore in detail the driving forces in LBL formation, interactions between adsorbed polyelectrolyte layers, and the limitations for LBL self-assembly of polyelectrolytes.

Key to controlling the LBL process and the multilayer film properties is understanding factors which govern the adsorption of polyions from solution onto surfaces containing oppositely charged polymers.  Extensive theoretical work regarding the adsorption criteria for neutral polymers onto surfaces has previously been done, which have given both quantitative predictions\cite{Fleer1982,Baumgartner1991} and scaling relations.\cite{DeGennes1976,Alexander1977}  For example, theoretical calculations for neutral polymers have shown that its adsorption is dependant on the chain length,\cite{Stuart1980} and these results have been confirmed by experiment.\cite{Stuart1980,Felter1970}  Theoretical attempts to predict polyelectrolyte adsorption are more recent, and have proven difficult to generalize to the formation of PEMs, since these assemblies are nonequilibrium.  In addition to accounting for complex electrostatic forces (i.e., by incorporating Debye-H\"uckel length-scales for polyions whose charge density can be highly sensitive to the local ionic strength at assembly),\cite{Chatellier1996} one must also consider thermodynamic and kinetic factors, which are both integral to the build-up of PEMs.\cite{Kovacevic2002} 

In this study, we examine the role of polyelectrolyte chain length in the adsorption of a simple model polyanion, poly(acrylic acid) onto silica particles capped with a model polycation, poly(allylamine hydrochloride).  Such studies of the adsorption behavior of charged macromolecules are important in order to understand various relevant biological surface and biointerface phenomena.  For instance, the adhesion of bacterial cells to solid surfaces (governed by electrostatic, van der Waals, and Lewis acid-base interactions) is often largely affected by cell-surface polymers, such as lipopolysaccharides.\cite{Jucker1997}  The affinity and reversibility of this adsorption process have been found to depend significantly on the molar mass of the cell surface polysaccharides involved in the adhesion.\cite{Jucker1997} 

\section{3. Methods}

\subsection{3.1 Experimental}

\subsubsection{3.1.1 Materials}

We used poly(allylamine hydrochloride) (PAH) as received from Polysciences, Inc. Poly(acrylic acid sodium salt) (PAA) of two different molecular weights ($M_{w}$ 2K and 450K) was purchased from Sigma-Aldrich and modified as described in section 5.3.2.  The two different molecular weights of PAA were chosen based on the large difference in the average chain length (i.e., by greater than 2 orders of magnitude).  Furthermore, dynamic light scattering characterization (Brookhaven BI9000; Thorn EMI Electron B2FBK RFI photomultiplier tube; Coherent Technologies Compass 315M-150 laser) of the low and high molecular weight PAA, showed very little overlap in the histograms of the hydrodynamic radius of the polymers under the solution conditions used for the adsorption experiments.  Chemical structures for the polycation, unmodified polyanion, 

%I don't know how to bruteforce this formatting issue above ( I am following the formatting given by the original manuscript
\begin{figure}[H]
    \begin{subfigure}[b]{0.3\textwidth}
        \includegraphics[scale=0.4]{fig1A.png}
        \caption{}
        \label{fig:A}
    \end{subfigure}
    \begin{subfigure}[b]{0.3\textwidth}
        \includegraphics[scale=0.25]{fig1B.png}
        \caption{}
        \label{fig:B}
    \end{subfigure}
    \begin{subfigure}[b]{0.3\textwidth}
        \includegraphics[scale=0.5]{fig1C.png}
        \caption{}
        \label{fig:C}
    \end{subfigure}
    \begin{subfigure}[b]{0.3\textwidth}
        \includegraphics[scale=0.5]{fig1D.png}
        \caption{}
        \label{fig:D}
    \end{subfigure}
    \begin{subfigure}[b]{0.3\textwidth}
        \includegraphics[scale=0.5]{fig1E.png}
        \caption{}
        \label{fig:E}
    \end{subfigure}
    \begin{subfigure}[b]{0.3\textwidth}
        \includegraphics[scale=0.5]{fig1E.png}
        \caption{}
        \label{fig:f}
    \end{subfigure}
    \caption{Structures of the polyelectrolytes used are: a. poly(allylamine hydrochloride), and b. poly(acrylic acid sodium salt).  Fluorophores used to label the polyanion are: c. 1-naphthalmethylamine, and d. dansyl ethylenediamine (D112).  Structures of labeled poly(acrylic acid) are: e. $NMA-PAA_{450K}$, and f. $Dan-PAA_{2K}$ }
    \label{figure 1}
\end{figure}

and the two labeled poly(acrylic acids) are presented in Figure 4.1.  We prepared aqueous polyelectrolyte solutions of concentration 10–2 M (based on the molecular weight of repeat units) using 18.2 M$\Omega\cdot$cm resistivity Millipore Milli-Q water and the adsorption experiments were done in the presence of 1 M NaCl.  

Two naphthalene-based fluorophores with amine functional groups, 5-dimethylaminonaphthalene-1-(N-(2-aminoethyl))sulfonamide (also known as dansyl ethylenediamine), and 1-naphthalmethylamine were purchased from Molecular Probes, and Sigma-Aldrich respectively.  Our adsorbing surface was colloidal silica of 70-100 nm in diameter, received from Nissan Chemical Industries. The pH of the solutions was adjusted to a value of 9.0 using NaOH.  At pH = 9.0, we can observe the adsorption of strongly charged PAA onto a mostly weakly charged PAH surface (since PAH was first adsorbed onto the colloidal particles).

\subsubsection{3.1.2 Derivization of PAA with Fluorophores}

Through a method previously described, \cite{weber1954fluorescent,Anghel1998} partial conversion of the carboxylic acid units of the 2K and 450K PAA to amide derivatives (amidization) was achieved in two separate reactions as follows:  Inside a dry three-neck flask attached to a condenser, vacuum dried PAA (2.0 g) was mixed with an anhydrous solvent, 1-methylpyrolidone (approximately 100 ml), in the presence of flowing nitrogen.  The mixture was stirred to allow for dissolution (in the case of 2K PAA) or additionally heated to $60^{\circ}$C for 2 hr (in the case of 450 K) to promote dissolution.  In a dry and nitrogen-rinsed flask, the fluorophore (7.0 $\times$ $10^{-4}$ mol) was dissolved in some 1-methylpyrolidone (approximately 5 ml) and injected into the PAA reaction pot.  Subsequently, dicyclohexylcarbodiimide (DCC, 8.0 $\times$ $10^{-4}$ mol), a coupling agent commonly used to promote amide formation was also injected into the reaction mixture upon dissolving in dry 1-methylphyrolidone (approximately 5 ml).  The use of an aprotic solvent, 1-methylphyrolidone, and DCC promotes the random covalent attachment of the hydrophobic fluorophores onto PAA.\cite{Anghel1998}  The final mixture was stirred in the dark at $60^{\circ}$C for 96 hr.  The reaction flask was then cooled in an ice bath, and concentrated NaOH was added dropwise to neutralize the solution, which resulted in precipitation of the sodium salt of the modified PAA.  The solid product was obtained by vacuum filtration and subsequently purified by twice precipitating an aqueous solution in methanol.  All solvents were removed to yield the final product by drying in a vacuum oven.

Both the unmodified and labeled PAA were characterized by $^{1}H$ NMR (300 MHz Varian Mercury) performed in $D_2O$.  In the case of 1-naphthalmethylamine-labeled PAA (NMA-$PAA_{450K}$) NMR spectrum analysis shows the expected ratio of naphthalene (7) and amide (1) protons at chemical shifts of 7-8 ppm while the alpha aromatic proton (1) appears at 4.7 ppm.  In unmodified PAA, only methylene protons from the backbone (1-2.3 ppm) appear in the NMR spectrum.  The NMR spectrum of dansyl ethylenediamine-labeled PAA (Dan-$PAA_{2K}$) also reveals the expected chemical shifts at 7.2-8.3 ppm from the aromatic protons of the fluorophore, and the ,  methylene protons (4) from the amide link at around 3.4 ppm.  Both UV-vis absorption (Varian Cary 300-Bio Spectrophotometer), and $^{1}H$ NMR spectral analysis of the final product suggested little contamination of labeled PAA and the degree of modification was calculated by integration of the $^{1}H$ NMR spectral peaks.  It was determined that PAA of 450 K was 3.8 mol\% modified while 4 mol\% of the 2 K PAA was labeled.  The degree of derivatization of PAA was kept at a low value to prevent interference from the hydrophobic fluorescent tags on the electrostatic adsorption of the polyelectrolyte.  Fluorescence intensity calibration plots at variable concentration of $NMA-PAA_{450K}$ and $Dan-PAA_{2K}$ were prepared using a FluoroMax-2 (Jobin Yvon-SPEX; slit width set to 2.5 cm). 

\subsubsection{3.1.3a Preliminary Adsorption of PAH onto Colloids}

An excess amount of the polycation layer, poly(allylamine hydrochloride), was first provided for adsorption onto colloidal $SiO_2$ according to the following standard protocol previously outlined for the preparation of PAH/PAA multilayers onto $SiO_2$.\cite{Burke2003}  In this protocol the polyelectrolyte adsorption conditions are set to a pH value of 9.0 and 1.0 M of NaCl was used to ensure highest coverage of the colloid with the polycation.  Colloids were dispersed in a polyelectrolyte solution for 30 min, followed by 1 h of centrifugation at 4500 rpm.  The supernatant is then removed and the colloids are washed three times with copious amounts of Milli-Q $H_2O$ (adjusted to pH = 9) to remove unadsorbed polymer.  Each wash step consists of 30 min of sonication (to expose colloids to wash solution while preventing aggregation), 1 h centrifugation, and finally extraction of the liquid.  

\subsubsection{3.1.3b Competitive Adsorption of PAA onto a PAH-Coated Surface}

The competitive adsorption of long (450 K) versus short (2 K) chains of PAA was examined by supplying in solution a stoichiometric surface coverage amount of each of $Dan-PAA_{2K}$ and $NMA-PAA_{450K}$ to 2.0 g/L of PAH-coated particles.  This required surface coverage amount of polyelectrolyte (3-5 $\times$ $10^{-4}$ M repeat units of each type of PAA) for 2.0 g/L of PAH-coated particles was determined from quantitative titrations of PAH/PAA-coated particles with poly(diallyldimethyl-ammonium chloride) (PDAC).\cite{Burke2003}   The molar concentration of labeled PAA necessary to cover the PAH surface (3-5 $\times$ $10^{-4}$ M for 2.0 g/L of particles) was obtained by converting the bare mass of titrant (PDAC) required to achieve charge overcompensation, and thus multilayer formation, to a molar surface coverage value (as determined by $\zeta$ potential measurements).\cite{Burke2003}   Specifically, this point refers to the mass of titrant at the initial plateau point after the half neutralization point in the titration curve (i.e., 0.05g/L of PDAC for 2.0 g/L of coated particles).  Although this estimation of the surface coverage value is based on a titration experiment involving a different polyelectrolyte than that used in our experiment, it provides a reasonable approximation of the required coverage amount.  Furthermore, the titration experiment was performed under comparable solution conditions to our PAA adsorption study (1mM NaCl and solution pH = 9) and in both cases the adsorbing polyelectrolyte is strongly charged.  Thus, assuming that the adsorption isotherms are comparable, the error in determining the surface coverage value is estimated to be in the range of the experimental uncertainty reported in the titration study, $\pm$ 9.6 \%.  We allowed 24 h to ensure coverage adsorption of the labeled PAA onto PAH since previous PAH/PAA multilayer studies have shown significant time-dependent adsorption at low polyelectrolyte concentrations (i.e., $<10^{-3}$ M).\cite{Mermut2003}

\subsubsection{3.2 Theoretical}

\subsubsection{3.2.1 Development of Mean-field lattice model}

It has been recognized that long polymers are
preferentially adsorbed compared to shorter polymers,
due to the short polymer's greater gain of
translational entropy per monomer in three dimensions
as compared to two 
%(see for example Sections 4.4.2.1 and 5.4.1 of G.J. Fleer et al)
\cite{Fleer1993}.In light of the small number of polymers on each
spherical surface, we adapt existing methods to a
fully discrete model to analyze this phenomenon.

We shall use a lattice model to describe our experimental situation.  
Lattices will appear in two ways:  (1) covering the surface of a sphere by a two-dimensional lattice, and
(2) filling space with a regular three-dimensional cubic lattice.  
The geometry of the lattices will not be important.  
%We shall use the simple cubic lattice for our three-dimensional lattice.
We shall assume that the two-dimensional lattice has coordination number $q$ 
(that is, each site on the surface has exactly $q$ neighbouring sites).  We assume that 
the distance between lattice sites corresponds to the distance between adjacent monomers within a 
polymer.  We assume that in equilibrium, every lattice site in the surface is covered by an adsorbed 
monomer.
%Changed all V1 to V, in order to avoid confusion of it being some sort of index, as used in epsilon_i later on 
Our experiment has a large number $n_{sphere}$ of spheres in a solution of volume $V_{sol}$.  To simplify 
our model and keep the concentrations the same, we shall consider only a single sphere in a solution
of volume $V:=V_{sol}/n_{sphere}$.  We shall refer to this as our model system.
We shall view this small three-dimensional region of the solution in our model system as being filled by a 
cubic lattice.
Let $V_{latt}$ be the number of lattice sites in our solution region of volume $V$.



%%Proceeded to sub in subsequent duplicate definition here below this section 
% We shall consider polymers of two different lengths, which we shall call ``short'' and ``long.''  Their lengths
% will be described by two positive integer parameters $W$ and $n$.

% \begin{verse}
% A \textbf{short polymer} will consist of $W$ monomers.  
% Let $n_S$ be the number of short polymers in the model system. 
% \\
% A \textbf{long polymer} will consist of $nW$ monomers.  
% Let $n_L$ be the number of long polymers in the model system. 
% \end{verse}

%after shifting the below definitions this off course was out of place as it was defining things too late.
We shall often refer to our short and long polymers as $W$-mers and $nW$-mers respectively.
We shall want the total mass of the two kinds of monomers in our system to be the same, so we 
shall assume that $n_S \,=\, n\cdot n_L$.

% this used to be in list form (latex verse format) but I linearized it as per Ozzy's request
% not sure if I should be using commas and if I should boldface the Values, Expressions
%also the parenthesis "(because..."  clashes with "n(a-j)"
We let the number of $nW$-mers in solution to be $n_L-j$, the number of $W$-mers adsorbed on the surface to be $n(a-j)$ (because there are $anW$ sites on the surface, all occupied, of which $jnW$ are occupied by $nW$-mers, leaving $anW-jnW$ to be  occupied by $W$-mers), and the number of $W$-mers in solution to be  $n_S-n(a-j)$.


% %commented this out and shifted up as well
% We shall often refer to our short and long polymers as $W$-mers and $nW$-mers respectively.
% We shall want the total mass of the two kinds of monomers in our system to be the same, so we 
% shall assume that $n_S \,=\, n\cdot n_L$.

%reversed the sequence, by starting with definition and then introducing term as per Ozzy
Let the number of lattice sites on the surface of one sphere be $A_{latt}$, and let the number of long polymers that would 
% removed exactly
%removed a from linear font, as it is introduced as an expression
fully cover all the sites of the surface be:
% we are labelling all equations as per Ozzy's request
\begin{equation}
    \label{eq.Number of Long Polymers that Fully Cover All Sites}
       a  \;:=\;  \frac{A_{latt}}{nW}
\end{equation}

Then $na$ is the number of short polymers that can completely cover the surface.
For simplicity, we assume that $a$ is an integer.

%%removed as this is something that shall be discussed in results
% We shall see that the underlying entropic reason that
% mostly long chains get adsorbed is that $V_{latt}\gg A_{latt}$, i.e.\ there are many more possible locations in 
% the three-dimensional solution than on the two-dimensional surface.


We shall model our polymer configurations in the solution in a slightly different way from those on the surface.
The difference in approach is due to the fact that the polymers are dense on the surface but 
dilute in the solution.  We shall model the polymer configurations in the solution as a collection of 
self-avoiding walks in the cubic lattice, with no interaction between walks (because of the dilute solution).
For the polymers adsorbed densely on the surface lattice, we shall use mean-field calculations 
of the Flory-Huggins type \cite{Flory1953}.
% (similar to the lattice approach in Park et al., Macromolecules 2001;
%(see also Section XII.1 of \cite{Flory1953}).

Recall that our ensemble consists of $n_S$ (short) $W$-mers and $n_L$ (long) $nW$-mers.
Of these, some are adsorbed onto the surface of the sphere, and the others float freely in the 
solution of volume $V$.  We write $j$ for the number of (long) $nW$-mers that are adsorbed 
onto the surface in a particular configuration of the system.

%%removed this and moved it to page 8 this may create issues in fluidity, will investigate further
%This implies that


% \begin{verse}
%  ($i$) the number of $nW$-mers in solution is $n_L-j$;
%  \\
%  ($ii$) the number of $W$-mers adsorbed on the surface is $n(a-j)$ (because there are $anW$ sites on the
%  surface, all occupied, of which $jnW$ are occupied by $nW$-mers, leaving $anW-jnW$ to be 
%  occupied by $W$-mers); and
%  \\
%  ($iii$) the number of $W$-mers in solution is $n_S-n(a-j)$.
%  \end{verse}  
We shall write $\mathcal{E}_j$ for the set of all configurations that have exactly $j$ $nW$-mers adsorbed 
onto the surface.  Observe that the possible values of $j$ are the integers from 0 to $a$.
The full space of configurations in our model is $\cup_{j=0}^a {\mathcal{E}_j}$, 
which we shall call $\mathcal{E}$.

Since the only energy in this model comes from the adsorption contacts between monomer and surface,
and since the number of such contacts is the same (namely $A_{latt}$) in every configuration
in $\mathcal{E}$, we see that the energy of every configuration of $\mathcal{E}$ is exactly the same.
Therefore, according to the Boltzmann distribution, every configuration in $\mathcal{E}$ is equally
% changed this sentence partially
likely. Thus to calculate the probabilities of events we count the configurations.
Specifically, $|\mathcal{A}|$ is the number of configurations in a set $\mathcal{A}$, and the probability that $j$ long polymers are adsorbed is $\frac{|\mathcal{E}_j|}{|\mathcal{E}|}$   where $(j=0,1,\ldots,a)$.

%% replace this with a linear text above
% we have
% \[
%     \hbox{Probability that $j$ long polymers are adsorbed}  \;=\; 
%     \frac{|\mathcal{E}_j|}{|\mathcal{E}|}\,,   \hspace{5mm}(j=0,1,\ldots,a).
% \]
Thus, to find the most likely number of $nW$-mers (and $W$-mers) to be adsorbed onto the sphere,
we need to find which of the sets $\mathcal{E}_j$ is largest.




\subsubsection{3.2.2 Ratio Approximation Formula}
%inserted a motivation that will explain why on page six we want to maximize epsilon j, also tried to flow into the first step of this modelling process being modelling the config of a single polymer
We now wish to estimate the size of each $\mathcal{E}_j$, in order to  determine the maximum $\mathcal{E}_j$,which is the preferential configuration of the experimental system. We begin by modelling the configuration of a single  polymer in a dilute solution by a self-avoiding walk (SAW) in a lattice, where a path in the lattice does not visit any site more than once.
%reordered this 
Let the number of SAWs that start from a specified lattice site (``the origin'') and visit a total of $\ell$ sites be represented as $c_{\ell}$. Additionally, let $\mu_3$ and $A_3$ be constants that depend on the choice of lattice, and $\gamma_3$ be a universal critical exponent that is the same for all three-dimensional lattices, their values are known to be approximately, \cite{Chen2002,Madras2013}  
%%not sure if I should incorporate this in the linear text too
%Their values are known to be approximately \cite{Chen2002,Madras2013}
\begin{equation}
   \label{eq.gammas}   \mu_3 \;=\;  4.684, \hspace{5mm}
        \gamma_3 \;=\;  1.162,    \hspace{5mm}\hbox{and}\hspace{5mm}
    A_3  \;=\;    0.2573  \,.  %1.205/4.683 \;=\;  \,.
\end{equation}

%Am I permitted to say parameters and given????
\noindent Given the above parameters, a fundamental property of SAWs is that the asymptotic behaviour of $c_{\ell}$ on the three-dimensional cubic lattice is 
\begin{equation}
    \label{eq.sawscale}
       c_{\ell}  \;\sim  \;  A_3 \, {\ell}^{\gamma_3-1}  \mu_3^{\ell}    \hspace{5mm}\hbox{as $\ell\rightarrow\infty$}.
\end{equation}
% (This has not been proven rigorously, but even mathematicians do not doubt its truth.) 
% Here $\mu_3$ and $A_3$ depend on the choice of lattice, but $\gamma_3$ is a universal critical exponent
% that is the same for all three-dimensional lattices.  
% Their values are known to be approximately \cite{Chen2002,Madras2013}
% \begin{equation}
%   \label{eq.gammas}   \mu_3 \;=\;  4.684, \hspace{5mm}
%         \gamma_3 \;=\;  1.162,    \hspace{5mm}\hbox{and}\hspace{5mm}
%     A_3  \;=\;    0.2573  \,.  %1.205/4.683 \;=\;  \,.
% \end{equation}
%(Estimation in \cite{Chen2002} is for the number of SAWs with $\ell$ \textit{steps} (i.e.\ $\ell+1$ \textit{sites}), which we write  $c_{\ell+1}=(A_3\mu_3)\ell^{\gamma_3-1}\mu_3^{\ell}$; reference \cite{Chen2002} obtains $A_3\mu_3=1.205$.)

%%Commented this out as it doesn't flow anymore given all derivations will be in the supplement.
%Since the number of sites in the lattice corresponding to the region of solution is $V_{latt}$, 
%we have $V_{latt}c_{\ell}$ ways to place a polymer of size ${\ell}$.  
%As we are neglecting the possibility of overlapping polymers in the dilute solution, it follows that the number of ways to 
%place $N$ identical $\ell$-mers is 
%\begin{equation}
%  \label{eq.Npoly}
%   \frac{(V_{latt}c_{\ell})^N}{N!}  \,.   
%\end{equation}

\smallskip

% For a particular choice of $j$, we want to know the number of ways to place $j$ $nW$-mers 
% and $a(n-j)$ $W$-mers on the surface, without overlapping, to fill all $anW$ lattice sites on the surface.
% This is a hard combinatorial problem, so we shall use the following mean-field (Flory-Huggins) approach.
% We shall place one polymer at a time, starting with the long ones.
% Let $\tilde{w}_k$ be the number of ways to place the $k^{th}$ $nW$-mer on the surface, given that
% $(k-1)$ $nW$-mers have already been placed.  
% To begin with, there are $anW-(k-1)nW$ available sites for the first monomer.  Recall that  $q$ is the number of
% neighbours of each site in the surface lattice.
% In the absence of other polymers, there would be $q$ choices for the second monomer in the chain,
% and $q-1$ choices for each monomer after that (here we are using the non-reversed walk model of a polymer
% instead of the fully self-avoiding model).  But the number of choices should on average be reduced 
% by the fraction of the surface that has already been covered.  
% Thus, when we are trying to place the second monomer of our chain, there are
% $anW-(k-1)nW-1$ unoccupied sites, so each site has 
% probability $[anW-(k-1)nW-1]/[anW]$ of being available.  
% Thus there are $q[anW-(k-1)nW-1]/[anW]$ choices for the second monomer.  
% Similarly, after $i$ monomers of the current chain have been placed ($i\geq 2$), the fraction of the
% surface that has been covered is $[anW-(k-1)nW-i]/[anW]$, so there are 
% $(q-1)[anW-(k-1)nW-i]/[anW]$ choices for the $(i+1)^{th}$ monomer in this chain.  
% We conclude that 
% \begin{eqnarray}
%   \tilde{w}_k  & = &   [anW-(k-1)nW]\,\times \,q\left(\frac{ anW-(k-1)nW-1}{anW}\right)    \,\times
%   \nonumber    \\
%   & &    \hspace{22mm}
%      \prod_{i=2}^{nW-1}(q-1)\left(\frac{ anW-(k-1)nW-i}{anW}\right)  
%       \nonumber \\
%  & = &    \frac{[anW-(k-1)nW]!}{[anW-knW]!} \,q\,  \frac{(q-1)^{nW-2}}{(anW)^{nW-1}}  
%      \nonumber    \\
%      & = &    \frac{[anW-(k-1)nW]!}{[anW-knW]!}   anW \frac{q}{(q-1)^2}\left(  \frac{q-1}{anW}\right)^{nW}.
%      \label{eq.wk2}
% \end{eqnarray}
% Similarly, let $\tilde{u}_{\ell}$ be the number of ways to place the $\ell^{th}$ $W$-mer on the surface, given that
% ${\ell}-1$ $W$-mers (or a total of $(\ell-1)W$ monomers) have already been placed.   The same 
% argument as for $\tilde{w}_k$ gives
% \begin{eqnarray}
%   \tilde{u}_{\ell}  
%  & = &    \frac{[anW-(\ell-1)W]!}{[anW-\ell W]!} \,q\,  \frac{(q-1)^{W-2}}{(anW)^{W-1}}  
%      \nonumber    \\
%      & = &    \frac{[anW-(\ell-1)W]!}{[anW-\ell W]!}   anW \frac{q}{(q-1)^2}\left(  \frac{q-1}{anW}\right)^{W}.
%      \label{eq.uell2}
% \end{eqnarray}

% To evaluate $|\mathcal{E}_j|$, we assume that the unadsorbed solution
% is dilute enough that we don't need to worry about mutual exclusion between the free floating
% polymers.   
% For $\mathcal{E}_j$, we have $j$ adsorbed $nW$-mers, $na-nj$ adsorbed $W$-mers, 
% $n_L-j$ desorbed $nW$-mers, and $n_S-n(a-j)$ desorbed $W$-mers.   Thus, recalling 
% Equations (\ref{eq.sawscale}) and  (\ref{eq.Npoly}), we have
% \begin{equation}
%     |\mathcal{E}_j|  
%       \; = \; \frac{ \Psi_L^{n_L-j} }{(n_L-j)!} \,
%           \frac{ \Psi_S^{n_S-n(a-j)} }{(n_S-n(a-j))!} \,
%           \frac{\tilde{w}_1\tilde{w}_2\cdots \tilde{w}_j}{j!}
%                      \,   \frac{\tilde{u}_{nj+1}\tilde{u}_{nj+2}\cdots \tilde{u}_{an}}{(n(a-j))!}
%                                  \nonumber   \\
%            & & \hspace{7mm}  \times \, \frac{\tilde{w}_1\tilde{w}_2\cdots \tilde{w}_j}{j!}
%                     \,   \frac{\tilde{u}_{nj+1}\tilde{u}_{nj+2}\cdots \tilde{u}_{an}}{(n(a-j))!}  % \,e^{-nW\epsilon/kT}
        \label{eq.Yj}
% \end{equation}
% where 
% \begin{equation}
%     \label{eq.psidef}   
%   \Psi_L\;=\;V_{latt} A_3 (nW)^{\gamma_3-1} \mu_{3}^{nW}   
%     \hspace{5mm}\hbox{and} \hspace{5mm}
%      \Psi_S \;=\; V_{latt}A_3W^{\gamma_3-1}\mu_3^W \,.
% \end{equation}




To determine which value of $j$ maximizes $|\mathcal{E}_j|$, we look at ratios of consecutive terms:
% \begin{eqnarray}
%     \frac{|\mathcal{E}_{j+1}|}{|\mathcal{E}_j|}   & = & 
%  \frac{
%      \left(\frac{\Psi_L^{n_{L}-j-1}}{(n_L-j-1)!}\right)\cdot\left(\frac{\Psi_S^{n_S-n(a-j-1)}}{
%     (n_S-n(a-j-1))!}\right) 
%       }{
%  \left(\frac{\Psi_L^{n_{L}-j}}{(n_L-j)!}\right)\cdot\left(\frac{\Psi_S^{n_S-n(a-j)}}{
%     (n_S-n(a-j))!}\right)  }  \,   \frac{\tilde{w}_{j+1} }{  \tilde{u}_{nj+1}\cdots \tilde{u}_{n(j+1)} } 
%     \nonumber \\
%     & & \hspace{42mm}  
%      \times   \, \frac{ j! \, (na-nj)!}{(j+1)!\, (na-nj-n)!}  
%     \nonumber    \\
%     & = & \frac{\Psi_S^n}{\Psi_L}
%   % \frac{ \left(  V_{latt} A_3 W^{\gamma_3-1}\right)^{n-1}}{n^{\gamma_3-1}}
%     \,   \frac{\tilde{w}_{j+1} }{  \tilde{u}_{nj+1}\cdots \tilde{u}_{n(j+1)} }   \,\left(  \frac{n_L-j}{j+1}\right)
%     \nonumber \\
%     & &   \hspace{8mm}  \times \;  \frac{(na-nj)\cdots (na-nj-n+1)}{(n_S-na+nj+1)\cdots(n_S-na+nj+n)} \,.
%         \label{eq.Yratio1}
% \end{eqnarray}
% By Equation (\ref{eq.psidef}), we have
% \begin{equation}
%   \label{eq.psiratio} 
%       \frac{\Psi_S^n}{\Psi_L}   \;=\;   \frac{ (V_{latt}\, A_3 \,W^{\gamma_3-1})^{n-1}}{n^{\gamma_3-1}}\,.
% \end{equation}
% From Equations (\ref{eq.wk2}) and (\ref{eq.uell2}), we find
% \begin{eqnarray}
%      \frac{\tilde{w}_{j+1} }{  \tilde{u}_{nj+1}\cdots \tilde{u}_{n(j+1)} }  & = & 
%         \left( \frac{(q-1)^2}{qanW}\right)^{n-1}   \times \,
%     \frac{   \frac{(anW-jnW)!}{(anW-(j+1)nW)!}  }{
%         \prod_{\ell=0}^{n-1} \frac{ (anW-(nj+\ell)W)!}{(anW-(nj+\ell+1)W)!}   }
%      \nonumber  \\
%      & = & \left( \frac{(q-1)^2}{qanW}\right)^{n-1}  \,\times\, 1.
%      \label{eq.Yratioa}     
% \end{eqnarray}
% Next we use the approximation   $(a+1)(a+2)\cdots (a+m-1) \,\approx \,(a+\frac{m}{2})^m$
% (essentially, replacing the geometric mean by the arithmetic mean) to obtain
% \begin{equation}
%       \frac{(na-nj)\cdots (na-nj-n+1)}{(n_S-na+nj+1)\cdots(n_S-na+nj+n)} 
%       \; \approx \;    \frac{ [na-(j+\frac{1}{2})n]^n }{ [n_S-na+nj+\frac{n}{2}]^n} \,.\hspace{3mm}
%   %     \nonumber  \\
%   %     & \approx & \left(  \frac{n(a-j-\frac{1}{2})}{n_S-(a-j-\frac{1}{2})}\right)^n  .
%       \label{eq.Yratiob}
% \end{equation} 
through the following approximation (derived in the supplement):
\begin{equation}
    \label{eq.Yratio2}
       \frac{|\mathcal{E}_{j+1}|}{|\mathcal{E}_j|} \; \approx \; 
       \left(  \frac{ V_{latt}A_3W^{\gamma_3-1}(q-1)^2}{aWq} \,
          \frac{(a-j-\frac{1}{2})}{n_S-n(a-j-\frac{1}{2})} \right)^{n+o(n)}   \,.
\end{equation}


The quantity on the right of Equation (\ref{eq.Yratio2}) decreases in $j$.  
% attempted to reword this, used j_min instead, so they know it is a fixed and not a new variable 
%added still in hopes of getting the point across but may sound childish\
%will have to replace j* in graphics in supplement possibly
%only concern is that min will throw people off, given that we are searching for a max....
Let $j_{fixed}$ be a fixed value that represents the the smallest available number of j sites occupied by nW-mers for which the right-hand side of Equation (\ref{eq.Yratio2}) is still less than 1.

Then we see that $|\mathcal{E}_{j+1}|/|\mathcal{E}_j|<1$ whenever $j\geq j_{fixed}$, and 
$|\mathcal{E}_{j+1}|/|\mathcal{E}_j|\geq 1$ whenever $j< j_{fixed}$.
That is, $|\mathcal{E}_j|$ is decreasing in $j$ when $j\geq j_{fixed}$, and 
increasing when $j< j_{fixed}$.
It follows that $|\mathcal{E}_j|$ is maximized at $j=j_{fixed}$, and that we can find $j_{fixed}$ by seeing 
where the right-hand side of Equation (\ref{eq.Yratio2}) is close to 1.   

%equation containers everywhere, unlabeled though, not sure what journal wants.
%moved the definition of K to prior in sequence, and made it an equation
\noindent We let, 
\begin{equation}
    \label{}
K \;=\;   \frac{ V_{latt} }{aW}\, \frac{   A_3W^{\gamma_3-1}(q-1)^2}{q} \,,
\end{equation}


%changed the text here slightly for flow
%added noindents to fix the format, not sure if that's allowed, either by journal or by best practices
\noindent Therefore $j^*$ satisfies,
\begin{equation}
    \label{}
    1  \;\approx \; 
     K  \,
     %\frac{ V_{latt}A_3W^{\gamma_3-1}(q-1)^2}{aWq} \,
          \frac{a-j^*-\frac{1}{2}}{n_S-n(a-j^*-\frac{1}{2})}    \,,  
             %\hspace{5mm}\hbox{where}\hspace{5mm}
             %K \;=\;   \frac{ V_{latt} }{aW}\, \frac{   A_3W^{\gamma_3-1}(q-1)^2}{q} \,,
\end{equation}

\noindent which can be rewritten as, 
\begin{equation}
    \label{}
a-j_{fixed}-\frac{1}{2} \; \approx\;    \frac{n_S}{K+n}   \,.
\end{equation}

%added new page as formatting was off
\newpage

We observe that $K\gg n$; indeed,  $V_{latt}/aWn \,=\,V_{latt}/A_{latt}\gg 1$.  
Thus we can replace $K+n$ by $K$ in the above, and omit the term $\frac{1}{2}$, resulting in the   
approximation 
\begin{equation}
    \label{eq.jstar}
     1-\frac{j_{fixed}}{a}     \; \approx   \; 
        \left(  \frac{n_S\,W}{V_{latt} }\right) \,\left(   \frac{q}{A_3W^{\gamma_3-1}(q-1)^2}\right)  \,.
\end{equation}
The above equation can be interpreted as follows.  The left hand side, $1-j_{fixed}/a$, is the fraction of the colloid sphere's surface that is covered by short monomers.  The ratio $n_SW/V_{latt}$ gives
the concentration of monomers corresponding to those appearing on the short chains only.
%The quantity in the second set of parentheses in Equation (\ref{eq.jstar}) cannot be large, since $q<(q-1)^2$ and $\gamma_3>1$.
If the solution is fairly dilute, then we see that the left side of (\ref{eq.jstar}) is small,
and hence only a small fraction of the surface is covered by short polymers.


%[\textit{Remark:}  Observe that $n_L$ does not appear in Equation (\ref{eq.jstar}).  
%In the expression for $|\mathcal{E}_{j+1}|/|\mathcal{E}_j|$,   only $n_S$ appears inside the 
%expression that is raised to the power $n$, while $n_L$ appears outside this expression and is
%raised to the power 1.]  

\section{4. Results and Discussion}

\subsection{4.1 Experimental}

\subsubsection{4.1.1 Fluorescence Calibration Plots of Labeled PAA in Solution}

As shown in Figure 4.2, we initially characterized the fluorescence emission intensity of the labels on the short- and long-chain PAA in solution.  Fluorescence spectra

\begin{figure}[H]
\includegraphics[scale=0.5]{fig2.png}
\caption{Fluorescence emission spectra obtained for variable concentration of: a. short-chain labeled PAA ($Dan-PAA_{2K}$), and b. long-chain labeled PAA ($NMA-PAA_{450K}$) in water.}
\label{figure 2}
\end{figure}

were acquired for $NMA-PAA_{450K}$ (by excitation at $\lambda_{Ex}$ = 290 nm; emission at $\lambda_{Em}$ = 340),\cite{Anghel1998} and $Dan-PAA_{2K}$ (by excitation at $\lambda_{Ex}$ = 335 nm; emission at $\lambda_{Em}$ = 550 nm)\cite{Bednar1985} in the concentration range of $10^{-1}$ to $10^{-5}$ g/L of PAA.  Note that the sharp peaks centered at twice the excitation wavelength, 580 nm (in the $NMA-PAA_{450K}$ sample) and 670 nm (in the $Dan-PAA_{2K}$ sample), are instrumental artefacts.  Specifically, they are scattered light transmitted as the second order diffraction of the emission monochromator. 

We compared the fluorescence intensity of the two fluorophore-labeled PAA samples, at the peak maximum.  Although both of the PAA samples were modified by approximately 4\% fluorophores, the fluorescence intensity exhibited by $NMA-PAA_{450K}$ was found to be greater by an order of magnitude than that of $Dan-PAA_{2K}$ prepared at an identical solution concentration.  The smaller fluorescence signal from the $Dan-PAA_{2K}$ is attributed to the dansyl fluorophore, which is known to fluoresce much less intensely in water as compared to nonpolar organic solutions.\cite{weber1954fluorescent,Bednar1985,Chen1983}  A fluorescence signal for both samples, however, was detectable at concentrations above the critical coverage adsorption concentration used for the study ($10^{-4}$ M).  As indicated in Figure 3, we also determined that there was no significant quenching of one fluorophore by the other, by obtaining the fluorescence spectrum of a mixed sample of both $NMA-PAA_{450K}$ and $Dan-PAA_{2K}$ in solution at identical concentrations of 5.0 $\times$ $10^{-3}$ g/L (above the coverage concentration).  From Figure 3, the relative ratio of the $NMA-PAA_{450K}$ to $Dan-PAA_{2K}$ emission peak in the mixed solution was determined as 13:1.  The relative fluorescence intensity of $NMA-PAA_{450K}$ to $Dan-PAA_{2K}$ obtained from isolated solutions was similar to that of the mixed PAA solution indicating that the emission of one fluorophore did not perturb the emission of the other by any measurable amount. (Figure 3).

\begin{figure}[H]
\includegraphics[scale=2.0]{fig3.png}
\caption{Pre-adsorbed fluorescence spectrum of a mixed solution containing 5.0 × 10–3 g/L of each of $NMA-PAA_{450K}$ ($\lambda_{Ex}$ = 290 nm), and $Dan-PAA_{2K}$ ($\lambda_{Ex}$ = 335 nm).  Emission intensity ratio of $NMA-PAA_{450K}$ to $Dan-PAA_{2K}$  is 13:1.}
\label{figure 3}
\end{figure}

\subsubsection{4.1.2 Competitive Adsorption of Labeled PAA onto a PAH Surface}

Successful adsorption of PAA onto a PAH coated surface is known from previous $\zeta$ potential measurements and solid-state NMR spectroscopy of similar multilayer systems.\cite{Burke2003,Smith2004}  We provided an equal surface coverage concentration of both the 2K and the 450K PAA repeat units to a quantitative amount of PAH layered $SiO_2$.  At concentrations lower than that required for complete coverage, both 2K and 450K PAA would adsorb to meet until surface coverage achieved, and such an experiment would fail to test for preferential adsorption.  If experiments are conducted at concentrations higher than that required for surface coverage, we might not observe a significant change in the 


\begin{figure}[H]
\includegraphics[scale=3.0]{fig4.png}
\caption{Possible extreme results for the competitive adsorption experiment of PAA onto PAH-coated particles.  The two extreme cases show preference of the surface for only the short chains (emission only at $\lambda_{Em}$ = 550 from adsorbed $Dan-PAA_{2K}$), or solely the long chains (emission only at $\lambda_{Em}$ = 550 from adsorbed $NMA-PAA_{450K}$).  The third intermediate case (not shown) is that of indifferent adsorption.}
\label{figure 4}
\end{figure}

fluorescence signal to identify preferential adsorption. Figure 4 illustrates the two extreme outcomes for this competitive adsorption study.  Given an identical concentration of repeat units, the two extreme cases involve either sole preference for shorter chains of PAA (i.e., Case 1, where fluorescence is only detected from $Dan-PAA_{2K}$) or the longer ones (i.e., Case 2, where only NMA-PAA450 fluorescence is observed).  A third possible scenario is that of an unbiased adsorption of both short and long chains, in which no change in the relative fluorescence signals would be expected before and after the adsorption. 

Although previous studies examining the preparation of PAH/PAA multilayers suggest adsorption times on the order of tens of minutes, (supplying excess concentration of the adsorbing polyelectrolyte), we allowed 24 h for the competitive adsorption study of PAA since here, we supply only a minimum coverage concentration of short and of long chains of PAA.  After 24 h, we examined the fluorescence of both the PAA/PAH-coated particles (thoroughly rinsed free of excess polymer), and the remaining unadsorbed PAA in the supernatant.  Figure 5 shows the relative fluorescence intensity obtained for the labeled PAA adsorbed on the particles while Figure 6 shows that of the unadsorbed PAA remaining in the supernatant.  After exciting both fluorophores on the particles, no detectable fluorescence emission from $Dan-PAA_{2K}$ at $\lambda_{Em}$ = 550 nm was observed.  However, the particles did exhibit a strong emission signal from at $\lambda_{Em}$ = 340 nm.  Similar inspection of the supernatant indicated an opposite trend, where we observed significant fluorescence from $Dan-PAA_{2K}$.  Furthermore the intensity ratio of NMA-PAA450: $Dan-PAA_{2K}$ in the supernatant was 3.4:1, which was significantly less than the original pre-adsorption ratio of 13:1.  Fluorescence analysis of both the PAA adsorbed 


\begin{figure}[H]
\includegraphics[scale=1.75]{fig5.png}
\caption{Fluorescence emission spectra obtained for variable concentration of: a. short-chain labeled PAA ($Dan-PAA_{2K}$), and b. long-chain labeled PAA ($NMA-PAA_{450K}$) in water.}
\label{figure 5}
\end{figure}


\begin{figure}[H]
\includegraphics[scale=1.75]{fig6.png}
\caption{Fluorescence emission spectra obtained for variable concentration of: a. short-chain labeled PAA ($Dan-PAA_{2K}$), and b. long-chain labeled PAA ($NMA-PAA_{450K}$) in water.}
\label{figure 6}
\end{figure}


onto the particles and the PAA remaining in the supernatant suggest the preferred adsorption of the higher molecular weight component PAA onto the PAH-coated colloidal particles after 24 hr.  Also, we do not expect the effects of PAA polydispersity to significantly mislead our results since: a) the two extreme chain lengths differ by two orders of magnitude, and b) preliminary DLS characterization of the 2 K and 450 K chains showed negligible overlap in the distribution of their hydrodynamic radius.

\subsubsection{4.1.3 General Processes in Adsorption}

To understand why there is preferential adsorption of longer chains of PAA onto PAH-coated particles, it is worthwhile to examine the general steps involved in the process of polymer adsorption.  The adsorption mechanism begins by the transport of bulk polymer to the surface-adsorbing site.  Kinetically, this diffusion-limited step would favor the arrival of shorter chains to the surface.  Transport is then followed by polymer attachment to the surface, and lastly rearrangements, which can occur in the adsorbing layer.  The two initial steps generally occur rapidly and the adsorption rate is usually transport-limited since the polymer arriving at the surface adsorbs immediately.\cite{Dijt1990} As surface coverage increases, adsorption becomes hindered.  At this stage, further adsorption is highly dependant on attachment to the remaining free sites.\cite{Hoogeveen1996}  At very high surface coverage and approaching maximum coverage, transport plays a minimal role and the specific attachment process becomes increasingly important.\cite{Hoogeveen1996} Rearrangement processes are also generally much slower and thus some polymer adsorption can be considered to be irreversible at relatively short time-scales.\cite{Cafe1982,Meadows1988}  Therefore, in the case where adsorption is examined at time-scales much larger than that required to achieve full surface coverage, kinetic contributions are less likely to influence the preferential adsorption of polymers and over much longer time-scales.  In this plateau region of an adsorption isotherm adsorption is less dependant on transport.\cite{Hoogeveen1996}



\subsubsection{4.1.4 Polyelectrolyte Adsorption}


In polyelectrolyte adsorption additional interactions need to be considered which have important thermodynamic implications. The theory of neutral polymer adsorption can be extended to polyelectrolyte adsorption by incorporating Debye-H\"uckel theory.\cite{Chatellier1996}  Polyelectrolytes adsorb onto oppositely charged surfaces when the energy of adsorption exceeds the net entropy resulting from the culmination of entropic losses (associated with the reduction in the number of configurations available to the polyelectrolyte), and entropic gains (associated with the release of counterions).  Whether or not adsorption will occur depends on if there is sufficient energy given to the system to overcome the entropy loss i.e. if the temperature is below a critical value to reduce the free energy and to drive towards adsorption.

Enthalpic contributions include the interaction type and strength between the polyion and the surface (i.e., electrostatic attraction with each link of the order $k_B$T) as well as the interaction between charged segments (i.e., electrostatic repulsion), which oppose adsorption.\cite{Hoogeveen1996,VonGoeler1994}  In the adsorption of charged polymers the surface charge is compensated when the adsorbed charge balances with the surface charge such that the electrostatic attraction of the segments with the surface is balanced by the repulsion of segments in the adsorbing layer.  In polyelectrolyte multilayer adsorption the surface charge is overcompensated, causing net electrostatic repulsion.  In achieving charge overcompensation, weakly charged polyelectrolytes differ from strongly charged polyelectrolytes in that more polyion molecules have to adsorb in order to overcompensate the surface charge and this is why more polyelectrolyte adsorption is generally observed.  Chain stiffness and conformation also have significant effect on adsorption, particularly in the case of weak polyelectrolytes.\cite{Dzubiella2003}  For example, adsorption of weak charged polyelectrolytes onto an oppositely charged surface in the form of “loops and tails” can be favored over a more flat, “train-like” configuration where electrostatic interactions between polyion segments and the surface are maximized. \cite{Borisov1994}  The conformation of polyelectrolyte adsorption is thus highly sensitive to the electrostatic environment during adsorption, for example, the solvent pH and ionic strength.\cite{Notley2004}  In the case of polyelectrolyte multilayer adsorption, nonelectrostatic short range forces such as hydrophobic interactions have also been observed, which enhance stability in adsorption.\cite{Kotov1999}  


\subsubsection{4.1.5 Long- versus Short-Chain Adsorption }

\subsubsection{4.1.5a Enthalpic Considerations}

In considering the preferential adsorption of PAA onto PAH-coated particles, we assume uniform capping of the $SiO_2$ with the polycation, PAH.  Thus all potential adsorption surface sites are assumed to be entirely composed of PAH repeat units, and the adsorption occurs on a chemically homogeneous surface.  This is a reasonable assumption for PAH adsorbed in excess concentration and under solution conditions that render it a weakly charged polycation (i.e., adsorption solution adjusted to pH = 9, near the pK of PAH).\cite{Burke2003,Smith2003}  This implies that all electrostatic attractions between identical polycations and polyanions are energetically equivalent (i.e., of equivalent $k_B$T, since in both cases each PAA sample contains an equal number of repeat units).\cite{Dubas1999}  Additionally, the enthaplic loss derived from displacement of counterions from their associated polyelectrolytes is also similar for the short and long chain systems.  Hence, if we assume that the electrostatics is one of the dominant driving forces in multilayer formation, we can simplify the thermodynamic comparison by presenting an ideal case in which the enthalpic energy is identical in the two cases, under identical solvent conditions (i.e., pH value and salt concentration). 

Interestingly, the addition of a high concentration of electrolyte has been shown to influence the adsorption of polyelectrolytes of varying chain lengths.  In previous adsorption studies of a strongly charged polyelectrolyte, poly(styrene sulfonate) onto a chemically homogeneous $Fe_2O_3$ surface, preferential adsorption was observed for shorter chains from a salt-free solution while longer chains were preferred in the presence of 0.1 M NaCl.\cite{Ramachandran1987,Ramachandran1988}  Adsorption studies of PAA,\cite{Wright1987} polyacrylate,\cite{Bain1982} and carboxymethyl cellulose\cite{Bain1982} adsorbed onto $BaSO_4$ report similar preferential adsorption of low molecular weight components in the absence of salt.  However, these adsorption isotherms reveal significant displacement of the low molecular weight with the high molecular weight in the presence of 0.5 M NaCl.\cite{Bain1982} This adsorption behavior is rationalized using a sequential adsorption process, which suggests that first, smaller chains are adsorbed.\cite{DeLaat1995}  For example, short PAA chains initially adsorbed eventually generate an electrostatic barrier from charge overcompensation occurring on the positive surface.  This barrier strongly affects the diffusion of chains towards the PAA covered surface.  Specifically, the barrier can prevent longer chains from accessing the surface, and thus limits their displacement of pre-adsorbed shorter chains.  With an increased salt concentration the barrier is lowered, permitting longer chains to reach the surface and adsorb.  At this point, the adsorption preference is shifted to longer chains as experimentally observed.\cite{DeLaat1995}  We therefore suggest that our observation of preferred adsorption of longer PAA chains onto PAH after a 24 h adsorption period is likely restricted to the time-window past which such displacement effects are likely to occur.  Further supporting evidence for the occurrence of short-chain displacement is given by recent adsorption experiments of model cationic oligomers onto colloids,\cite{Shin2001} and short polyions assembled onto proteins,\cite{Houska2004} which suggest that shorter chains have more difficulty forming loops and tails under assembly conditions where the polyion is weakly ionized.  As such, adsorbed short-chain polyions can be more easily displaced by longer chains from failure to make a sufficient number of ionic contacts.  

\subsubsection{4.1.5b Entropic Considerations}

Although the LBL self-assembly technique is based on electrostatic attraction of positively and negatively charged polyions, the primary driving force is presumed to be entropy and not enthalpy.  In the electrostatic assembly of multilayers, the condensation of oppositely charged polyions liberates low molar mass counterions.  This process increases the entropy of the system, comparable to the polyelectrolyte complexes formed in solution.\cite{Kabanov1994,Philipp1989}  Thus, although other interactions such as hydrophobic interactions, charge transfer interactions, - stacking forces, or H-bonding may also contribute, the successful formation of polyelectrolyte multilayers is primarily attributed to the entropy gain from ion 
exchange.\cite{Kotov1999,Bertrand2000}  

In determining the effect of entropy on the preferential adsorption of short versus long chains of polyelectrolytes, three entropic contributions need to be considered.  First is the net entropy associated with the liberation of the counterions.  Since we provide an identical number of repeat units for both long- and short-chain PAA, and assume coverage adsorption in both cases, the counterion release entropy should be similar for both 2 K and 450 K.  Secondly, we need to compare the configurational entropy of the short- versus long-chain polyelectrolyte upon adsorption.  As the polyelectrolyte chain length is increased, the entropy penalty associated with the adsorption becomes greater.  This is because there are more configurational restrictions to surface-bound long chains as compared to adsorbed short chains.  The loss in configurational entropy upon adsorption is therefore expected to be much larger for the 450 K PAA, which would favor short-chain adsorption.  Lastly, we need to compare the configurational entropy of the free polyelectrolytes in solution.  There is a far greater entropic gain from having more shorter chains in solution, which can explore a greater number of configurational states than a fewer longer chain species in solution.  Similar to the liberation of counterions which drives the LBL assembly process, the entropy gain of having more free short chains in solution favors the preferential adsorption of long-chain PAA onto the PAH surface.  Experimentally, the configurational entropy difference between short and long-chains of PAA in solution appears to be the governing factor leading to the preferential adsorption of 450 K PAA.  Moreover, this argument is supported by reports of shorter-chain polyelectrolyte displacement in exchange for adsorption of longer-chains on Fe2O3 and BaSO4 in the presence of salt, as previously mentioned.\cite{DeLaat1995}  Analysis of both the fluorescence spectrum of the PAA-adsorbed particles, and that of the remaining supernatant after the adsorption, suggest that the preferential adsorption of long- over short-chain PAA is dominated by the entropy gain of keeping short chains free in solution.

\subsection{4.2 Theoretical}

\subsubsection{4.2.1 Validation of Model to Experimental Data}

Figure 7 demonstrates the decreasing behaviour of the ratio of probabilities $\left(\frac{\mathcal{E}_{j+1}}{\mathcal{E}_j}\right)$, as described by the mean field lattice model. Using the experimental data we observe the ratio of probabilities approaching one, from functional values greater than one, corresponding to $\mathcal{E}_j$ approaching a maximum value.


\begin{figure}[H]
\includegraphics[scale=0.50]{fig11.png}
\caption{Ratio of Probabilities}
\label{figure 11}
\end{figure}

(Spyros)
We proceed to bridge the model based on the the ratio of configurations $\mathcal{E}_{j}$ to that of preferential entropy values, using a derivation of the Boltzmann entropy expression from the ratio of configurations (derivation in supplement), and then graphed the experimental data using the derived expression for entropy. As depicted in Figure 8, there is a decreasing, yet positive rate of change of S for the incremental addition of j long chains on the colloid surface. Not only is there a positive entropy increase as one adds additional long chain Polymers to the surface, but it is also steadily decreasing the increments of entropy change as more and more long chains are added to the surface, suggestive of entropy approaching a plateau, or more specifically, a maximum value. 


% this figure is within the range of kT's (4.11×10−21J) but the value that fluctuates is 6 orders less, not sure if I should omit completely
\begin{figure}[H]
\includegraphics[scale=0.6]{fig12.png}
\caption{Change in Entropy}
\label{figure 12}
\end{figure}

Our experiment has shown a clear preference for short chains to remain in the solution, to that of the surface of the colloid, because the preferential pathway of the system is to remove the obstruction of long chains in the solution, as this would vastly increase the freedom  of short chains in solution and thus remove any constraints on the entropic ranges of the short chains. Therefore, it is believed that long chains are forced to the surfaces of the colloids, given that it is the only path that would not restrict the short chains. This is now also demonstrated by the model, because we observe that as one adds more and more long chains to the colloid the probability of that occurrence of specific number of long chains increases. Therefore, from the perspective of the colloid surface, the probability of having maximal coverage by long chains is greater than the probability of full coverage by short chains.

(end of Spyros)


\section{Conclusions}

\bibliography{LongShort}

\end{document}
